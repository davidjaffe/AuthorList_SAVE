%-------------------------------------------------------------------------------
% This file contains the title, author and abstract.
% It also contains all relevant document numbers used by the different cover pages.
%-------------------------------------------------------------------------------
%\usepackage{\ATLASLATEXPATH atlasphysics}
%\usepackage{\ATLASLATEXPATH atlasheavyion}
% Title
\AtlasTitle{Measurement of multi-particle azimuthal correlations in $pp$, $p$+Pb and low-multiplicity Pb+Pb collisions with the ATLAS detector}

% Author - this does not work with revtex (add it after \begin{document})
\author{The ATLAS Collaboration}


% CERN preprint number
 \PreprintIdNumber{CERN-EP-2017-048}

% ATLAS date - arXiv submission; to be filled in by the Physics Office
% \AtlasDate{\today}

% arXiv identifier
% \arXivId{14XX.YYYY}

% HepData record
% \HepDataRecord{ZZZZZZZZ}

% Submission journal and final reference
 \AtlasJournal{EPJC}
% \AtlasDOI{}

% Abstract - % directly after { is important for correct indentation
\AtlasAbstract{%
Multi-particle cumulants and corresponding Fourier harmonics are measured for azimuthal angle distributions of charged particles in \pp collisions at \sqs = 5.02 and 13~TeV and in \pPb collisions at \sqn = 5.02~TeV, and compared to the results obtained for low-multiplicity \PbPb collisions at \sqn = 2.76~TeV. These measurements aim to assess the collective nature of particle production. The measurements of multi-particle cumulants confirm the evidence for collective phenomena in \pPb and low-multiplicity \PbPb collisions.  On the other hand, the \pp results for four-particle cumulants do not demonstrate collective behaviour, indicating that they may be biased by contributions from non-flow correlations.  A comparison of multi-particle cumulants and derived Fourier harmonics across different collision systems is presented as a function of the charged-particle multiplicity. For a given multiplicity, the measured Fourier harmonics are largest in \PbPb, smaller in \pPb and smallest in \pp collisions. The \pp results show no dependence on the collision energy, nor on the multiplicity.  
}

